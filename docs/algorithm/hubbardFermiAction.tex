\documentclass[a4paper, fleqn, twoside, notitlepage]{scrartcl}
\KOMAoptions{captions=tableheading,
             toc=bibliography
           }

\usepackage[a4paper,
            vdivide={3.3cm,,3.3cm},
            hdivide={2cm,,2cm}
            ]{geometry}

\usepackage{ucs}                % more unicode
\usepackage[utf8x]{inputenc}
\usepackage[T1]{fontenc}
\usepackage[english]{babel}

\usepackage{amsmath}
\usepackage{amsfonts}
\usepackage{amssymb}
\usepackage{amsthm}
\usepackage{mathtools}
\usepackage{commath}            % for nicer differentials
\usepackage{bm}                 % bold math
\usepackage{dsfont}

\usepackage{datetime}
\usepackage{authblk}
\usepackage{picinpar}           % picture in paragraph
\usepackage{graphics}           % addition to above
\usepackage{float}              % place graphics with "H"
\usepackage{caption}
\usepackage{subcaption}
\usepackage{cite}
\usepackage{placeins}           % FloatBarrier
\usepackage[dvipsnames,hyperref]{xcolor}
\usepackage[colorinlistoftodos]{todonotes}
\usepackage[pdftex, ocgcolorlinks]{hyperref}
\usepackage{cleveref}

%----------------------------------------------------------
% general info
\newdateformat{isodate}{\THEYEAR-\twodigit{\THEMONTH}-\twodigit{\THEDAY}}

\newcommand{\theauthor}{Jan-Lukas Wynen}
\newcommand{\theinstitute}{IAS-4}
\newcommand{\thetitle}{Algorithms for Fermion action in Hubbard model}
\newcommand{\thedate}{\isodate\today}


\author[1]{\theauthor}
\affil[1]{Institute for Advanced Simulation 4\\
Forschungszentrum J\"ulich, Germany}

\title{\thetitle}
\date{\thedate|\currenttime}
% \institute{\theinstitute}

%----------------------------------------------------------
% colours
\definecolor{rwthblau}     {RGB}{0,84,159}
\definecolor{rwthbordeaux} {RGB}{161,16,53}
\definecolor{rwthgelb}     {RGB}{255,237,0}
\definecolor{rwthgrun}     {RGB}{87,171,39}
\definecolor{rwthlila}     {RGB}{122,111,172}
\definecolor{rwthmaigrun}  {RGB}{189,205,0}
\definecolor{rwthmagenta}  {RGB}{227,0,102}
\definecolor{rwthorange}   {RGB}{246,168,0}
\definecolor{rwthpetrol}   {RGB}{0,97,101}
\definecolor{rwthrot}      {RGB}{204,7,30}
\definecolor{rwthschwarz}  {RGB}{0,0,0}
\definecolor{rwthturkis}   {RGB}{0,152,161}
\definecolor{rwthviolett}  {RGB}{97,33,88}

\hypersetup{
  colorlinks=true,
  pdftitle={\thetitle},
  pdfauthor={\theauthor},
  linkcolor=rwthblau,
  citecolor=rwthrot,
  urlcolor=rwthturkis,
  linktoc=all   % put links on chapter names and page numbers in toc
}

%----------------------------------------------------------
% abbreviations
\newcommand{\unit}[1]{\,\text{#1}}
\newcommand{\ev}{\,\text{eV}}
\newcommand{\kev}{\,\text{keV}}
\newcommand{\mev}{\,\text{MeV}}
\newcommand{\gev}{\,\text{GeV}}

\let\nodoti\i
\renewcommand{\i}{\mathrm{i}}
\renewcommand{\epsilon}{\varepsilon}

\renewcommand{\Re}{\text{Re}\,}
\renewcommand{\Im}{\text{Im}\,}

%----------------------------------------------------------
% miscellaneous
% remove trailing dot from number
\renewcommand*{\figureformat}{%
  \figurename~\thefigure%
}
\renewcommand*{\tableformat}{%
  \tablename~\thetable%
}

% text to be displayed at end of proof
\renewcommand{\qedsymbol}{q.e.d.}

% fallback if hyperref is not included
\providecommand{\url}[1]{\texttt{#1}}

% make "Table" and "Figure" in captions bold
\captionsetup{labelfont=bf}

%%% Local Variables:
%%% mode: latex
%%% TeX-master: "hubardFermiAction"
%%% End:



\begin{document}

\maketitle
\tableofcontents
\clearpage

\section{Definitions}

The Fermionic action is
\begin{align}
  S_\text{ferm} = - \log \det M(\phi, \tilde{\kappa}, \tilde{\mu}) M(-\phi, \sigma_{\tilde{\kappa}}\tilde{\kappa}, -\tilde{\mu}).\label{eq:ferm_action}
\end{align}
where $\phi$ is the auxiliary field that is integrated over. In general $\sigma_{\tilde{\kappa}} = -1$ but for bipartite lattices, a particle-hole transformation can be used to get $\sigma_{\tilde{\kappa}} = +1$. The fermion matrix is
\begin{align}
  {M(\phi, \tilde{\kappa}, \tilde{\mu})}_{x't';xt}
  &= (1+\tilde{\mu})\delta_{x'x}\delta_{t't} - \kappa_{x'x}\delta_{t't} - \mathcal{B}_{t'} e^{\i\phi_{xt}}\delta_{x'x}\delta_{t'(t+1)}\\
  &\equiv K_{x'x}\delta_{t't} - \mathcal{B}_{t'}{(F_{t'})}_{x'x}\delta_{t'(t+1)}.
\end{align}
where
\begin{align}
  K_{x'x} &= (1+\tilde{\mu})\delta_{x'x} - \kappa_{x'x},\\
  {(F_{t'})}_{x'x} &= e^{\i\phi_{x(t'-1)}}\delta_{x'x}.
\end{align}
Note that the second $M$ in~\eqref{eq:ferm_action} can not be expressed as $M^*(\phi)$ even for bipartite lattices and $\mu=0$ because $\phi$ is potentially complex valued.
Anti-periodic boundary conditions are encoded by
\begin{align}
  \mathcal{B}_t =
  \begin{cases}
    +1,\quad 0 < t < N_t\\
    -1,\quad t = 0
  \end{cases}
\end{align}
and periodicity in Kronecker deltas.
In block matrix form, $M$ is a cyclic lower block bidiagonal matrix:
\begin{align}
  M(\phi, \tilde{\kappa}, \tilde{\mu}) =
  \begin{pmatrix}
    K    &      &        &        & F_0 \\
    -F_1 & K    &        &        &     \\
         & -F_2 & K      &        &     \\
         &      & \ddots & \ddots &     \\
         &      &        &-F_{N_t-1}&K   \\
  \end{pmatrix}.\label{eq:ferm_mat_block}
\end{align}
The parameters are
\begin{itemize}
\item $\delta = \beta / N_t$ and $\beta$ the inverse temperature
\item $\tilde{\kappa} = \delta\kappa$ and $\kappa$ the hopping matrix
\item $\tilde{\mu} = \delta\mu$ and $\mu$ the chemical potential
\end{itemize}


\section{Determinant of $M$}

The determinant of $M$ is computed using an LU-decomposition. The decomposition can be calculated analytically in terms of spacial matrices given the specific structure in equation~\eqref{eq:ferm_mat_block}.
Use the following ansatz\footnote{This is an adaptation of the algorithm presented in \url{https://hrcak.srce.hr/100527}}:
\begin{align}
  L =
  \begin{pmatrix}
    1   &     &    &        &        &\\
    l_0 & 1   &    &        &        &\\
        & l_1 & \ddots &        &        &\\
        &     & \ddots & 1      &        &\\
        &     &   & l_{n-3} & 1      &\\
        &     &   &        & l_{n-2} & 1
  \end{pmatrix},
  \; U =
  \begin{pmatrix}
    d_0 &     &      &   &        & v_0\\
        & d_1 &     &    &        & v_1\\
        &     & d_2 &    &        & \vdots \\
        &     &     & \ddots &        & v_{n-3} \\
        &     &     &    & d_{n-2} & v_{n-2} \\
        &     &     &    &        & d_{n-1}
  \end{pmatrix}
\end{align}
Multiplying this out and comparing sides of the equation $M = LU$ leads to a set of recursive equations:
\begin{itemize}
\item $d_i = K$ for $0 \le i \le N_t-2$;\hspace{2em} $d_{N_t-1} = K - l_{N_t-2}v_{N_t-2}$
\item $l_i d_i = -F_{i+1}$ for $0 \le i \le N_t-2$
\item $v_0 = F_0$;\hspace{2em} $l_{i-1} v_{i-1} + v_i = 0$ for $1 \le i \le N_t-2$
\end{itemize}
Since we only care about the determinant and
\begin{align}
  \det M = \det L \det U = 1 \prod_{i=0}^{N_t-1}\,d_i
\end{align}
we only need the $d$'s.
Resumming ('remultiplying'?) the equations for the non-trivial $d$ gives
\begin{align}
  d_{N_t-1} = K + F_{N_t-1}K^{-1} F_{N_t-2}K^{-1} \cdots F_{1}K^{-1} F_{0}.
\end{align}
Hence the (log) determinant is
\begin{align}
  \log \det M &= \log \big({(\det(K))}^{N_t-1} \det (K + F_{N_t-1}K^{-1} F_{N_t-2}K^{-1} \cdots F_{1}K^{-1} F_{0})\big)\\
              &= N_t \log \det(K)  + \log \det (1 + K^{-1}F_{N_t-1}K^{-1} F_{N_t-2}K^{-1} \cdots F_{1}K^{-1} F_{0})\\
              &\equiv N_t \log \det(K)  + \log \det (1 + A)\label{eq:def_A}
\end{align}


\end{document}