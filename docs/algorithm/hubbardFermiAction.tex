\documentclass[a4paper, fleqn, twoside, notitlepage]{scrartcl}
\KOMAoptions{captions=tableheading,
             toc=bibliography
           }

\usepackage[a4paper,
            vdivide={3.3cm,,3.3cm},
            hdivide={2cm,,2cm}
            ]{geometry}

\usepackage{ucs}                % more unicode
\usepackage[utf8x]{inputenc}
\usepackage[T1]{fontenc}
\usepackage[english]{babel}

\usepackage{amsmath}
\usepackage{amsfonts}
\usepackage{amssymb}
\usepackage{amsthm}
\usepackage{mathtools}
\usepackage{commath}            % for nicer differentials
\usepackage{bm}                 % bold math
\usepackage{dsfont}

\usepackage{datetime}
\usepackage{authblk}
\usepackage{picinpar}           % picture in paragraph
\usepackage{graphics}           % addition to above
\usepackage{float}              % place graphics with "H"
\usepackage{caption}
\usepackage{subcaption}
\usepackage{cite}
\usepackage{placeins}           % FloatBarrier
\usepackage[dvipsnames,hyperref]{xcolor}
\usepackage[colorinlistoftodos]{todonotes}
\usepackage[pdftex, ocgcolorlinks]{hyperref}
\usepackage{cleveref}

%----------------------------------------------------------
% general info
\newdateformat{isodate}{\THEYEAR-\twodigit{\THEMONTH}-\twodigit{\THEDAY}}

\newcommand{\theauthor}{Jan-Lukas Wynen}
\newcommand{\theinstitute}{IAS-4}
\newcommand{\thetitle}{Algorithms for Fermion action in Hubbard model}
\newcommand{\thedate}{\isodate\today}


\author[1]{\theauthor}
\affil[1]{Institute for Advanced Simulation 4\\
Forschungszentrum J\"ulich, Germany}

\title{\thetitle}
\date{\thedate|\currenttime}
% \institute{\theinstitute}

%----------------------------------------------------------
% colours
\definecolor{rwthblau}     {RGB}{0,84,159}
\definecolor{rwthbordeaux} {RGB}{161,16,53}
\definecolor{rwthgelb}     {RGB}{255,237,0}
\definecolor{rwthgrun}     {RGB}{87,171,39}
\definecolor{rwthlila}     {RGB}{122,111,172}
\definecolor{rwthmaigrun}  {RGB}{189,205,0}
\definecolor{rwthmagenta}  {RGB}{227,0,102}
\definecolor{rwthorange}   {RGB}{246,168,0}
\definecolor{rwthpetrol}   {RGB}{0,97,101}
\definecolor{rwthrot}      {RGB}{204,7,30}
\definecolor{rwthschwarz}  {RGB}{0,0,0}
\definecolor{rwthturkis}   {RGB}{0,152,161}
\definecolor{rwthviolett}  {RGB}{97,33,88}

\hypersetup{
  colorlinks=true,
  pdftitle={\thetitle},
  pdfauthor={\theauthor},
  linkcolor=rwthblau,
  citecolor=rwthrot,
  urlcolor=rwthturkis,
  linktoc=all   % put links on chapter names and page numbers in toc
}

%----------------------------------------------------------
% abbreviations
\newcommand{\unit}[1]{\,\text{#1}}
\newcommand{\ev}{\,\text{eV}}
\newcommand{\kev}{\,\text{keV}}
\newcommand{\mev}{\,\text{MeV}}
\newcommand{\gev}{\,\text{GeV}}

\let\nodoti\i
\renewcommand{\i}{\mathrm{i}}
\renewcommand{\epsilon}{\varepsilon}

\renewcommand{\Re}{\text{Re}\,}
\renewcommand{\Im}{\text{Im}\,}

%----------------------------------------------------------
% miscellaneous
% remove trailing dot from number
\renewcommand*{\figureformat}{%
  \figurename~\thefigure%
}
\renewcommand*{\tableformat}{%
  \tablename~\thetable%
}

% text to be displayed at end of proof
\renewcommand{\qedsymbol}{q.e.d.}

% fallback if hyperref is not included
\providecommand{\url}[1]{\texttt{#1}}

% make "Table" and "Figure" in captions bold
\captionsetup{labelfont=bf}

%%% Local Variables:
%%% mode: latex
%%% TeX-master: "hubardFermiAction"
%%% End:



\begin{document}

\maketitle
\tableofcontents

\vfill
\noindent
This document describes the algorithms used for the fermionic action in the Hubbard model.
They are implemented by these classes:
\begin{itemize}
\item \texttt{HubbardFermiMatrix}
\item \texttt{HubbardFermiAction}
\end{itemize}
and can be found int the following files:
\begin{itemize}
\item \texttt{cnxx/hubbardFermiMatrix.[hpp/cpp]}
\item \texttt{cnxx/hubbardFermiAction.[hpp/cpp]}
\end{itemize}
For usage information, see the source documentation of the classes.

\clearpage
\section{Definitions}

The Fermionic action is
\begin{align}
  S_\text{ferm} = - \log \det M(\phi, \tilde{\kappa}, \tilde{\mu}) M(-\phi, \sigma_{\tilde{\kappa}}\tilde{\kappa}, -\tilde{\mu}).\label{eq:ferm_action}
\end{align}
where $\phi$ is the auxiliary field that is integrated over. In general $\sigma_{\tilde{\kappa}} = -1$ but for bipartite lattices, a particle-hole transformation can be used to get $\sigma_{\tilde{\kappa}} = +1$. The fermion matrix is
\begin{align}
  {M(\phi, \tilde{\kappa}, \tilde{\mu})}_{x't';xt}
  &= (1+\tilde{\mu})\delta_{x'x}\delta_{t't} - \tilde{\kappa}_{x'x}\delta_{t't} - \mathcal{B}_{t'} e^{\i\phi_{xt}}\delta_{x'x}\delta_{t'(t+1)}\\
  &\equiv {K(\tilde{\kappa}, \tilde{\mu})}_{x'x}\delta_{t't} - \mathcal{B}_{t'}{F_{t'}(\phi)}_{x'x}\delta_{t'(t+1)}.
\end{align}
where
\begin{align}
  {K(\tilde{\kappa}, \tilde{\mu})}_{x'x} &= (1+\tilde{\mu})\delta_{x'x} - \tilde{\kappa}_{x'x},\\
  {F_{t'}(\phi)}_{x'x} &= e^{\i\phi_{x(t'-1)}}\delta_{x'x}.
\end{align}
Note that the second $M$ in~\eqref{eq:ferm_action} can not be expressed as $M^*(\phi)$ even for bipartite lattices and $\mu=0$ because $\phi$ is potentially complex valued.
Anti-periodic boundary conditions are encoded by
\begin{align}
  \mathcal{B}_t =
  \begin{cases}
    +1,\quad 0 < t < N_t\\
    -1,\quad t = 0
  \end{cases}
\end{align}
and periodicity in Kronecker deltas.
In block matrix form, $M$ is a cyclic lower block bidiagonal matrix:
\begin{align}
  M =
  \begin{pmatrix}
    K    &      &        &        & F_0 \\
    -F_1 & K    &        &        &     \\
         & -F_2 & K      &        &     \\
         &      & \ddots & \ddots &     \\
         &      &        &-F_{N_t-1}&K   \\
  \end{pmatrix}.\label{eq:ferm_mat_block}
\end{align}
The parameters are
\begin{itemize}
\item $\delta = \beta / N_t$ and $\beta$ the inverse temperature
\item $\tilde{\kappa} = \delta\kappa$ and $\kappa$ the hopping matrix
\item $\tilde{\mu} = \delta\mu$ and $\mu$ the chemical potential
\end{itemize}


\section{Determinant of $M$}

The determinant of $M$ is computed using an LU-decomposition. The decomposition can be calculated analytically in terms of spacial matrices given the specific structure in equation~\eqref{eq:ferm_mat_block}.
Use the following ansatz\footnote{This is an adaptation of the algorithm presented in \url{https://hrcak.srce.hr/100527}}:
\begin{align}
  L =
  \begin{pmatrix}
    1   &     &    &        &        &\\
    l_0 & 1   &    &        &        &\\
        & l_1 & \ddots &        &        &\\
        &     & \ddots & 1      &        &\\
        &     &   & l_{n-3} & 1      &\\
        &     &   &        & l_{n-2} & 1
  \end{pmatrix},
  \; U =
  \begin{pmatrix}
    d_0 &     &      &   &        & v_0\\
        & d_1 &     &    &        & v_1\\
        &     & d_2 &    &        & \vdots \\
        &     &     & \ddots &        & v_{n-3} \\
        &     &     &    & d_{n-2} & v_{n-2} \\
        &     &     &    &        & d_{n-1}
  \end{pmatrix}
\end{align}
Multiplying this out and comparing sides of the equation $M = LU$ leads to a set of recursive equations:
\begin{itemize}
\item $d_i = K$ for $0 \le i \le N_t-2$;\hspace{2em} $d_{N_t-1} = K - l_{N_t-2}v_{N_t-2}$
\item $l_i d_i = -F_{i+1}$ for $0 \le i \le N_t-2$
\item $v_0 = F_0$;\hspace{2em} $l_{i-1} v_{i-1} + v_i = 0$ for $1 \le i \le N_t-2$
\end{itemize}
Since we only care about the determinant and
\begin{align}
  \det M = \det L \det U = (\prod_{i=0}^{N_t-1}\,1) (\prod_{i=0}^{N_t-1}\,d_i)
\end{align}
we only need the $d$'s.
Resumming ('remultiplying'?) the equations for the non-trivial $d$ gives
\begin{align}
  d_{N_t-1} = K + F_{N_t-1}K^{-1} F_{N_t-2}K^{-1} \cdots F_{1}K^{-1} F_{0}.
\end{align}
Hence the determinant is
\begin{align}
  \det M &= \big({(\det(K))}^{N_t-1} \det (K + F_{N_t-1}K^{-1} F_{N_t-2} \cdots K^{-1}F_{1}K^{-1} F_{0})\big)\\
         &= {(\det\,K)}^{N_t} \det(1 + K^{-1}F_{N_t-1}K^{-1} F_{N_t-2} \cdots K^{-1}F_{1}K^{-1} F_{0})
\end{align}
Apply the logarithm to get the final result:
\begin{resultbox}
  \vspace{-\baselineskip}
  \begin{align}
    \log \det M &= N_t \log \det(K)  + \log \det (1 + A),\label{eq:det_M}\\
    A &\equiv K^{-1}F_{N_t-1}K^{-1} F_{N_t-2} \cdots K^{-1}F_{1}K^{-1} F_{0}.\label{eq:def_A}
  \end{align}
\end{resultbox}
\noindent Note that the action~\eqref{eq:ferm_action} is the sum of contributions from particles and holes. In general those contributions are not related by a simple equation meaning that both need to be computed using the full expression~\eqref{eq:det_M}.
  
\section{Force}

The fermionic force at spacetime point $\mu\tau$ is given by
\begin{align}
  {(\dot{\pi}_{\text{ferm}})}_{\mu\tau} = -\dpd{H_{\text{ferm}}}{\phi_{\mu\tau}} = -\dpd{S_\text{ferm}}{\phi_{\mu\tau}} =  \dpd{}{\phi_{\mu\tau}} \Big[\log \det M(\phi, \tilde{\kappa}, \tilde{\mu}) + \log \det M(-\phi, \sigma_{\tilde{\kappa}}\tilde{\kappa}, -\tilde{\mu})\Big].
\end{align}
$\dot{\pi}$ is complex valued in general. For now we will ignore this fact and compute the force from the above equation.
Using~\eqref{eq:det_M} and Jacobi's formula, the derivative can be expressed as
\begin{align}
  {(\dot{\pi}_{\text{ferm}})}_{\mu\tau} = \Tr \Big[{(1+A_p)}^{-1}\dpd{}{\phi_{\mu\tau}}A_p + {(1+A_h)}^{-1}\dpd{}{\phi_{\mu\tau}}A_h\Big],
\end{align}
where
\begin{align}
  A_p = A(\phi, \tilde{\kappa}, \tilde{\mu}), \quad  A_h = A(-\phi, \sigma_{\tilde{\kappa}}\tilde{\kappa}, -\tilde{\mu}).
\end{align}
From now on, $A$ without explicit arguments refers to $A(\sigma_\phi\phi, \sigma_{\tilde{\kappa}}\tilde{\kappa}, \sigma_{\tilde{\mu}}\tilde{\mu})$ standing for either particle or hole and similarly for $K$ and $F$.\\

\noindent
Using~\cite{henderson:1980} ${(B+C)}^{-1} = C^{-1} - C^{-1}B{(1+C^{-1}B)}^{-1}C^{-1}$ with $B = 1$ and $C = A$, we write
\begin{align}
  \Tr {(1+A)}^{-1}\dpd{}{\phi_{\mu\tau}}A \;=\; \underbrace{\Tr A^{-1}\dpd{}{\phi_{\mu\tau}}A}_{(\text{I})} - \underbrace{\Tr A^{-1}{(1+A^{-1})}^{-1}A^{-1}\dpd{}{\phi_{\mu\tau}}A}_{(\text{II})}.
\end{align}
Calculate the first term. (The expression for $A^{-1}$ is given below in equation~\eqref{eq:def_Ainv})
\begin{align}
  (\text{I}) &= \underbrace{\color{highl1}{(F_0^{-1}K \cdots F_\tau^{-1}K)}_{xa'}}_{(\text{i})} {(F_{\tau+1}^{-1}K)}_{a'a} \underbrace{\color{highl2}{(F_{\tau+2}^{-1}K \cdots F_{N_t-1}^{-1}K)}_{ax'}}_{(\text{ii})}\nonumber\\
             &\quad\times \underbrace{\color{highl2}{(K^{-1}F_{N_t-1} \cdots K^{-1}F_{\tau+2})}_{x'b}}_{(\text{ii})} \Big[\dpd{}{\phi_{\mu\tau}}{(K^{-1}F_{\tau+1})}_{bb'}\Big] \underbrace{\color{highl1}{(K^{-1}F_\tau \cdots K^{-1}F_0)}_{b'x}}_{(\text{i})}\\
             &= {\color{highl1}\delta_{a'b'}^{(\text{i})}}{\color{highl2}\delta_{ab}^{(\text{ii})}} {(F_{\tau+1}^{-1}K)}_{a'a} \i \sigma_{\phi} \delta_{b'\mu} {(K^{-1}F_{\tau+1})}_{bb'}\\
             &= \i\sigma_\phi
\end{align}
Thus the sum of (I) for particles and holes vanishes because ${(\i\sigma_\phi)}_p + {(\i\sigma_\phi)}_h = \i - \i = 0$.
Treat the second term in a similar way:
\begin{align}
  (\text{II}) &= {(1+A^{-1})}^{-1}_{x'y}\; {(F_0^{-1}K \cdots F_\tau^{-1}K)}_{ya'} {(F_{\tau+1}^{-1}K)}_{a'a} \underbrace{\color{highl1}{(F_{\tau+2}^{-1}K \cdots F_{N_t-1}^{-1}K)}_{ay'}}_{(\text{i})}\nonumber\\
              &\quad\times \underbrace{\color{highl1}{(K^{-1}F_{N_t-1} \cdots K^{-1}F_{\tau+2})}_{y'b}}_{(\text{i})} \Big[\dpd{}{\phi_{\mu\tau}}{(K^{-1}F_{\tau+1})}_{bb'}\Big] \underbrace{\color{highl2}{(K^{-1}F_\tau \cdots K^{-1}F_0)}_{b'x}}_{(\text{ii})}\\
              &\quad\times \underbrace{\color{highl2}{(F_0^{-1}K \cdots F_\tau^{-1}K)}_{xc'}}_{(\text{ii})} {(F_{\tau+1}^{-1}K)}_{c'c} {(F_{\tau+2}^{-1}K \cdots F_{N_t-1}^{-1}K)}_{cx'}\nonumber\\
              &= {(1+A^{-1})}^{-1}_{x'y}\; {(F_0^{-1}K \cdots F_\tau^{-1}K)}_{ya'} \underbrace{{(F_{\tau+1}^{-1}K)}_{a'a} {\color{highl1}\delta_{ab}^{(\text{i})}} \i \sigma_\phi \delta_{b'\mu} {(K^{-1}F_{\tau+1})}_{bb'}}_{\i\sigma_\phi\delta_{a'b'}\delta_{b'\mu}}\nonumber\\[-\baselineskip]
              &\quad\times {\color{highl2}\delta_{b'c'}^{(\text{ii})}} {(F_{\tau+1}^{-1}K)}_{c'c} {(F_{\tau+2}^{-1}K \cdots F_{N_t-1}^{-1}K)}_{cx'}\\[1ex]
              &= {(1+A^{-1})}^{-1}_{x'y}\; {(F_0^{-1}K \cdots F_\tau^{-1}K)}_{ya'} \Big[- \dpd{}{\phi_{\mu\tau}}{(F_{\tau+1}^{-1}K)}_{a'c}\Big] {(F_{\tau+2}^{-1}K \cdots F_{N_t-1}^{-1}K)}_{cx'}\\
              &= - \Tr {(1+A^{-1})}^{-1} \dpd{}{\phi_{\mu\tau}}A^{-1}
\end{align}
The minus sign comes from the minus in $F^{-1}_{t'x} = e^{-\i\phi_{x(t'-1)}}\delta_{x'x}$ and cancels the minus in front of the whole term.\\
Combining those results gives the momentum as
\begin{resultbox}
  \vspace{-\baselineskip}
  \begin{align}
    {(\dot{\pi}_{\text{ferm}})}_{\mu\tau} &= \Tr \Big[{(1+A_p^{-1})}^{-1}\dpd{}{\phi_{\mu\tau}}A_p^{-1} + {(1+A_h^{-1})}^{-1}\dpd{}{\phi_{\mu\tau}}A_h^{-1}\Big],\\
    A^{-1} &\equiv F_0^{-1}K F_1^{-1}K \cdots F_{N_t-1}^{-1}K\label{eq:def_Ainv}
  \end{align}
\end{resultbox}
\noindent Hence it is possible to swap out $A$ for $A^{-1}$ in $\dot{\pi}_{\text{ferm}}$. This improves performance and possibly stability because all constituents ($F^{-1}$ and $K$) are sparse and only inverses of diagonal matrices need to be taken. $A$ on the other side contains $K^{-1}$ which is not trivial to compute and in general dense.\\

\noindent
For reference, write out $\dot{\pi}_{\text{ferm}}$ explicitly. First, in order to improve readability denote partial products of $A^{-1}$ as
\begin{align}
  A^{-1}_{tt'} \equiv F_t^{-1}K \cdots F_{t'}^{-1}K, \qquad A^{-1}_{tt} \equiv F_t^{-1}K.
\end{align}
The general case:
\begin{align}
  {(\dot{\pi}_{\text{ferm}})}_{\mu\tau} = -\i {\big(A^{-1}_{p,(\tau+1)(N_t-1)} {(1+A_p^{-1})}^{-1} A^{-1}_{p,0\tau}\big)}_{\mu\mu} + \i {\big(A^{-1}_{h,(\tau+1)(N_t-1)} {(1+A_h^{-1})}^{-1} A^{-1}_{h,0\tau}\big)}_{\mu\mu}
\end{align}
Cases new the boundary:
\begin{align}
  {(\dot{\pi}_{\text{ferm}})}_{\mu0} &= -\i {\big(A^{-1}_{p,1(N_t-1)} {(1+A_p^{-1})}^{-1} A^{-1}_{p,00}\big)}_{\mu\mu} + \i {\big(A^{-1}_{h,1(N_t-1)} {(1+A_h^{-1})}^{-1} A^{-1}_{h,00}\big)}_{\mu\mu}\\
  {(\dot{\pi}_{\text{ferm}})}_{\mu(N_t-1)} &= -\i {\big(A^{-1}_{p} {(1+A_p^{-1})}^{-1}\big)}_{\mu\mu} + \i {\big(A^{-1}_{h} {(1+A_h^{-1})}^{-1}\big)}_{\mu\mu}\\
  {(\dot{\pi}_{\text{ferm}})}_{\mu(N_t-2)} &= -\i {\big(A^{-1}_{p,(N_t-1)(N_t-1)} {(1+A_p^{-1})}^{-1} A^{-1}_{p,0(N_t-2)}\big)}_{\mu\mu} \nonumber\\
                                 &\quad + \i {\big(A^{-1}_{h,(N_t-1)(N_t-1)} {(1+A_h^{-1})}^{-1} A^{-1}_{h,0(N_t-2)}\big)}_{\mu\mu}
\end{align}
Keep in mind that $\mu$ is \emph{not} summed over, even when it appears twice in one term.


\clearpage
\bibliographystyle{abbrv}
\bibliography{references}

\end{document}